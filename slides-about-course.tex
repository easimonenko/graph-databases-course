% slides-about-course.tex
% About course
% Language: Russian
% Author: Evgeny Simonenko <easimonenko@mail.ru>
% License: CC BY-ND 4.0

\documentclass[11pt]{beamer}

\usepackage{polyglossia}
\setdefaultlanguage{russian}
\setotherlanguage{english}
\defaultfontfeatures{Ligatures={TeX},Renderer=Basic}
\setmainfont{FreeSerif}
\setsansfont{FreeSans}
\setmonofont[SizeFeatures={Size=10}]{FreeMono}

\title[Графовые СУБД]{Графовые базы данных}
\subtitle{О курсе}
\author[]{Симоненко Евгений}
\institute[]{Университет ИТМО}
\date[]{Санкт-Петербург, 2022}

\begin{document}

\begin{frame}
  \titlepage
\end{frame}

\begin{frame}
  \frametitle{Содержание}
  \tableofcontents
\end{frame}

\section{Об авторе курса}

\begin{frame}
  \frametitle{Об авторе курса}
  \begin{itemize}
  \item Изучает научную коммуникацию в Университете ИТМО
  \item В интересах (кроме графовых баз данных):
    \begin{itemize}
    \item GNU Emacs, Lisp, Scheme
    \item Haskell, Erlang, Standard ML
    \item Linux, NetBSD
    \item Embedded Systems, Robotics
    \end{itemize}
  \item По образованию математик
  \item По профессии программист и преподаватель
  \item Пишет на Хабре \url{https://habr.com/ru/users/easimonenko/posts/}
  \item Ведёт свой блог \url{https://easimonenko.github.io/}
  \end{itemize}
\end{frame}

\section{О курсе}

\begin{frame}
  \frametitle{О чём этот курс}
  \begin{itemize}
  \item Обсудим графовую модель данных и её отличия от реляционной.
  \item Посмотрим на ассортимент графовых СУБД.
  \item Установим, настроим, запустим и попробуем Neo4j.
  \item Научимся создавать и редактировать графы на языке Cypher.
  \item Научимся писать запросы к графам на языке Cypher.
  \item Узнаем как разрабатывать приложения Neo4j примере Node.js.
  \end{itemize}
\end{frame}

\section{Как проходят занятия}

\begin{frame}
  \frametitle{Как проходят занятия}
  \begin{itemize}
  \item Слушаем и обсуждаем
  \item Пробуем и тренеруемся
  \item Делимся опытом, задаём вопросы
  \item Между занятиями делаем зачётное задание
  \end{itemize}
\end{frame}

\section{Финальное задание}

\begin{frame}
  \frametitle{Финальное задание}
  На выбор один из нескольких вариантов:
  \begin{itemize}
  \item Подобрать предметную область, создать для неё графовую модель,
    подготовить пример набора данных, создать на его основе граф в СУБД Neo4j,
    провести анализ полученного графа.
  \item Изучить альтернативную графовую СУБД, рассмотреть вопросы из данного курса
    (администрирование, интерфейс пользователя, язык запросов, возможности).
  \item Внедрить в свой программный проект графовую СУБД (в качестве основной СУБД
    или для решения специфических задач).
  \item Изучить программный интерфейс к графовой СУБД для выбранного языка
    программирования (изучить все основные аспекты, такие как подключение к СУБД,
    отправка запросов, расшифровка ответов).
  \end{itemize}
  Возможны также различные сочетания предложенных вариантов.
\end{frame}

\begin{frame}
  \frametitle{Финальное задание}
  Порядок сдачи зачёта:
  \begin{itemize}
  \item Сделать выбор варианта задания.
  \item Собрать и изучить необходимую информацию.
  \item Поработать над выбранным вариантом.
  \item Предоставить артефакты выполненного задания.
  \item Получить обратную связь.
  \item Получить зачёт.
  \end{itemize}
\end{frame}

\begin{frame}
  \frametitle{Что почитать}
  \begin{itemize}
  \item Робинсон Я., Эифрем Э., Вебер Дж. Графовые базы данных. Новые возможности
    для работы. -- Пер. с англ. -- М.: ДМК-Пресс, 2016. -- 256 с.
  \item Эрик, Р. Семь баз данных за семь недель. Введение в современные базы данных
    и идеологию NoSQL. -- Пер. с англ. -- М.: ДМК Пресс, 2013. — 384 с.
  \end{itemize}
\end{frame}

\section*{Благодарность}

\begin{frame}
  \center
  \textit{Благодарю за внимание!}
  
  \textbf{\textsl{\inserttitle}}

  \insertauthor

  \url{mailto:easimonenko@mail.ru}

  \insertinstitute
\end{frame}

\end{document}
