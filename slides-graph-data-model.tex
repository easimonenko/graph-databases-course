% slides-graph-data-model.tex
% Graph Data Model
% Language: Russian
% Author: Evgeny Simonenko <easimonenko@mail.ru>
% License: CC BY-ND 4.0

\documentclass[11pt]{beamer}

\usepackage{polyglossia}
\setdefaultlanguage{russian}
\setotherlanguage{english}
\defaultfontfeatures{Ligatures={TeX},Renderer=Basic}
\setmainfont{FreeSerif}
\setsansfont{FreeSans}
\setmonofont[SizeFeatures={Size=10}]{FreeMono}

\title[Графовые СУБД]{Графовые базы данных}
\subtitle{Графовая модель данных}
\author[]{Симоненко Евгений}
\institute[]{Университет ИТМО}
\date[]{Санкт-Петербург, 2022}

\begin{document}

\begin{frame}
  \titlepage
\end{frame}

\begin{frame}
  \frametitle{Содержание}
  \tableofcontents
\end{frame}

\section{Модельный пример}

\begin{frame}
  \frametitle{Модельный пример}
  \begin{itemize}
  \item Вася дружит с Петей
  \item Петя подписан на Любу
  \item Люба слушает Анюту
  \item Анюта замужем за Лёшей
  \item Лёша живёт в Питере
  \item Вася учится в ИТМО
  \item Петя родился в Мичуринске
  \item Люба работает Кунсткамере
  \item ...
  \end{itemize}
\end{frame}

\begin{frame}
  \frametitle{Сущности модельного примера}
  \begin{itemize}
  \item \emph{Люди} (имеют имя, дату рождения):
    \begin{itemize}
    \item Вася
    \item Петя
    \item Люба
    \item Анюта
    \item Лёша
    \end{itemize}
  \item \emph{Города} (имеют название, местоположение, год основания):
    \begin{itemize}
    \item Питер
    \item Мичуринск
    \end{itemize}
  \item \emph{Место учёбы} (имеют название, местоположение, год основания):
    \begin{itemize}
    \item ИТМО
    \end{itemize}
  \item \emph{Место работы} (имеют название, местоположение, год основания):
    \begin{itemize}
    \item Кунсткамера
    \end{itemize}
  \end{itemize}
\end{frame}

\begin{frame}
  \frametitle{Отношения между сущностями}
  \begin{itemize}
  \item дружит c (друг)
  \item подписан на (следит за)
  \item слушает
  \item замужем за / женат на
  \item живёт в
  \item учится в
  \item родился в
  \item работает в
  \end{itemize}
\end{frame}

\section{Реляционная модель}

\begin{frame}
  \frametitle{Реляционная модель}
  Таблицы (списки кортежей):
  \begin{itemize}
  \item Люди (Peoples)
  \item Города (Cities) (оно же Место рождения (BirthPlace))
  \item Место работы (Workplace)
  \item Место учёбы (StudyPlace)
  \item Дружит с (FriendsWith)
  \item Подписан на (Following)
  \item Слушает (ListeningTo)
  \item Замужем за / женат на (MarriedTo)
  \item Живёт в (LivingIn)
  \item Учится в (StudiesIn)
  \item Родился в (BornIn)
  \item Работает в (WorksIn)
  \end{itemize}
\end{frame}

\section{Графовая модель}

\begin{frame}
  \frametitle{Графовая модель}
  \begin{itemize}
  \item Узлы (Nodes): предсталяют сущности. Могут хранить любые данные сущностей.
  \item Отношения (Relationships): представляют связи между узлами.
    Могут связывать любые сущности, и тоже хранить данные.
  \end{itemize}
  Например, Вася учится в ИТМО. Есть два узла: Вася и ИТМО. У узла Вася есть данные
  о его дате рождения, а у узла ИТМО есть данные о дате его основания.
  И есть связь между Васей и ИТМО, хранящая время обучения Васи в ИТМО.
\end{frame}

\section{Графовая модель vs Реляционная модель}

\begin{frame}
  \frametitle{Графовая модель vs Реляционная модель}
  \begin{itemize}
  \item В реляционной для каждого класса сущностей нужна отдельная таблица.
  \item В графовой сущности являются узлами, структура данных которых заранее не определяется.
  \item В реляционной для связей между сущностями используются ключи и вспомогательные таблицы
    (в пределе для каждого класса связей создаётся отдельная таблица).
  \item В графовой связи никак не классифицируются.
  \end{itemize}
\end{frame}

\section{Задание}

\begin{frame}
  \frametitle{Задание}
  \begin{itemize}
  \item Выделите в своей предметной области классы сущностей и классы отношений между ними.
  \item Подготовьте модельный пример.
  \item Смоделируйте его с помощью реляционного и графового подходов.
  \end{itemize}
\end{frame}

\section*{Благодарность}

\begin{frame}
  \center
  \textit{Благодарю за внимание!}
  
  \textbf{\textsl{\inserttitle}}

  \insertauthor

  \url{mailto:easimonenko@mail.ru}

  \insertinstitute
\end{frame}

\end{document}
